\documentclass{article}

\title{Xyz}


\begin{document}

\maketitle

\begin{abstract}
dfsd
\end{abstract}

\section{Introduction}
The gist of this paper is to discuss various augmentation approaches and study their effect in sentiment analysis by following two approaches.

Approach 1 - augmenting the dataset before training a classifier

Approach 2 - Augmenting the dataset while training a model.

We compare these approaches to DA on two different datasets.

\section{Model Descriptions}
\subsection{Problem Setting}
\subsection{Overview of the model}
\subsection{Data Preprocessing}
Spacy Tokenization \cite{srinivasa2018natural}
\subsection{Encoding Layer}

\section{Experiments}
\subsection{Experiment Settings}
Introduce parameter settings here
word embedding is 300D pretrained
adam
dropout for bilstm
dropout for wordembeddings
layers
epochs
Learning rate 
length of the sentences
\subsection{Dataset - Statistics}
\subsection{Implementation Details}
\subsection{Model Comparison}



body

\begin{thebibliography}{10}

\bibitem{srinivasa2018natural} Srinivasa-Desikan, Bhargav. Natural Language Processing and Computational Linguistics: A practical guide to text analysis with Python, Gensim, spaCy, and Keras. Packt Publishing Ltd, 2018.

\end{thebibliography}
\end{document}