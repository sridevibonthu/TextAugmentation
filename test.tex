
\documentclass{article}
\usepackage{multirow}
\title{Augmented NLP deep models for sentiment analysis  (or)  The effectiveness of text augmentation in sentiment analysis using deep learning}


\begin{document}

\maketitle

\begin{abstract}
This paper explores and campare multiple solutions to the problem of data augmentation in sentiment analysis.
This paper investigates two approaches to textual data augmentation for sentiment classification, offline and the other is online data modification. 
Offline means changing the data before the training is started and online data modification is using transformed samples during the training process. 
We are planning to find the best strategy for augmentation in terms of training time, accuracy, loss. 
We use 2 neural network architectures, 3 data augmentation methods, and test them on 2 different datasets. 
Our experiments indicate that the approach one improves the results ….


Keywords :  Data Augmentation, sentiment analysis, Back translation,


\end{abstract}

 

\section{Introduction}

Data Augmentation is ubiquitous in Vision, but very rare and disorganized in NLP. why?

Data augmentation helps to boost the performance of the model as it can be achieved through more data.
Image data augmentation is a standard practice in Computer Vision tasks, whereas Text data augmentation is rare in NLP tasks, due to the challenges it involves.
Rotation, flipping, contrast setting are not changing the semantics of the Image. An image of a bird is still an image of bird after applying augmentations.

Challenges with text augmentation..
1.
2.
3. 

The gist of this paper is to discuss various augmentation approaches and study their effect in sentiment analysis by following two approaches.

Approach 1 - augmenting the dataset before training a classifier - increase in training data

Approach 2 - Augmenting the dataset while training a model. - aug on batches

We compare these approaches to DA on two different datasets.
\section{Related Work}
Discussion on Augmentation strategies.

Data Augmentation in NLP is severly under-searched. When comes to Text Augmentation, researchers have many questions like Is it better to augment with language models? embeddings? or a thesarusus? How many words from the source text should be replaced? and How many data points need to be added to the training set?

Eventhough less work happened so far in text augementation, different authors leveraged augmentation to tickle NLP tasks via generating more text data to boost up the models.
some of them are 




NLP Data Augmentation Techniques - https://amitness.com/2020/05/data-augmentation-for-nlp/
1. Lexical Substitution - tries to substitute words present in a text without changing the meaning of the sentence.
1. Thesaurus-based substitution
2. word-embeddings substitution
3. masked language model
4. TF-IDF based word replacement

2. Back translation

3.Text Surface Transformation

4. Random Noise Injection
1. spelling error injection
2. qwerty keyboard error injection
3. unigram noising
4. blank noising
5, sentence shuffling
6. random insertion
7/ random swap
8. random deletion


5. Instance crossover augmentatio

6. syntax tree manipulation
7. mixup for text
a. word mix up
b. sent mix up

\section{Model Descriptions}
Mathematical models of LSTM, GRU, RNN
\subsection{Problem Setting}
\subsection{Overview of the model}
All the above models are optimized using Back Propagation Through Time (BPTT)[?] to minimize the following objective function.
\begin{equation}
a + b = b +a
\end{equation}
where.... is the ground truth
\subsection{Data Preprocessing}
Spacy Tokenization \cite{srinivasa2018natural}
\subsection{Encoding Layer}

\section{Experiments}
\subsection{Experiment Settings}
Introduce parameter settings here
word embedding is 300D pretrained
adam
dropout for bilstm
dropout for wordembeddings
layers
epochs
Learning rate 
length of the sentences
\subsection{Dataset - Statistics}
Sample REcords

EDA - number of examples, frequency distribution for length
\subsection{Implementation Details}
\subsection{Model Comparison}

\section{Results}
% Please add the following required packages to your document preamble:
% \usepackage{multirow}
\begin{table}[]
\begin{tabular}{cllll}
\multicolumn{1}{l}{}                  &           & RNN & LSTM & GRU \\
\multirow{2}{*}{Approach-1}           & Dataset 1 &     &      &     \\
                                      & Dataset 2 &     &      &     \\
\multirow{2}{*}{Approach-1}           & Dataset 1 &     &      &     \\
                                      & Dataset 2 &     &      &     \\
\multirow{2}{*}{Without Augmentation} & Dataset 1 &     &      &     \\
                                      & Dataset 2 &     &      &    
\end{tabular}
\end{table}

body

\begin{thebibliography}{10}

\bibitem{srinivasa2018natural} Srinivasa-Desikan, Bhargav. Natural Language Processing and Computational Linguistics: A practical guide to text analysis with Python, Gensim, spaCy, and Keras. Packt Publishing Ltd, 2018.

\end{thebibliography}
\end{document}